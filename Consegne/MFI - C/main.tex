\documentclass{article}
\usepackage{graphicx} % Required for inserting images

\title{MFI - C}
\author{Luca Barra}
\date{November 2023}

\usepackage{prooftree}
\hbadness=10000
\hfuzz=200pt

\newtheorem{definition}{Definizione}[section]
\newtheorem{fact}[definition]{Fact}
\newtheorem{theorem}[definition]{Teorema}
\newtheorem{corollary}[definition]{Corollario}
\newtheorem{lemma}[definition]{Lemma}
\newtheorem{proposition}[definition]{Propositione}
\newtheorem{remark}[definition]{Osservazione}
\newtheorem{exercise}[definition]{Esercizio}
\newtheorem{example}[definition]{Esempio}

%		\prooftree
%.                    A
%                    \quad 
%                    B
%		\justifies
%                    C
%.            \using \textit{rule}
%		\endprooftree
%
%Produce
%                            A     B
%                           ---------- rule
%                                C
% Ad A e B si possono sostituire altre occorrenze di \prooftree ... \endprooftree per ottenere una derivazione


\title{{\bf Consegna 1}}
\date{Novembre 2023}

\begin{document}\maketitle

\section{Esercizi su riscrittura, deduzione naturale e $\lambda$-calcolo con tipi semplici}


\begin{exercise}\label{ex:riscrittura}{\em
Si consideri il seguente sistema di regole di riscrittura per il calcolo simbolico della derivata $D_X$ in $X$
per espressioni aritmetiche costruite usando $+$, $*$ le indeterminate $X,Y$ (trattate come costanti) e le costanti $0,1$:
\[\begin{array}{lrcl}
(R1)   &   D_X(X) & \to & 1 \\
(R2)   &   D_X(Y) & \to & 0 \\
(R3)   &   D_X(u + v) & \to & D_X(u)  + D_X(v)  \\
(R4)   &   D_X(u * v) & \to & (u * D_X(v)) + (D_X(u) + v) \\
\end{array}\]

\noindent
Un esempio di riduzione in questo sistema \`e il seguente:
\[\begin{array}{llll}
D_X(X + Y) & \to & D_X(X) + D_X(Y) & \mbox{per $R3$} \\
                   & \to & 1 + D_X(Y) & \mbox{per $R1$} \\
                   & \to & 1 + 0 & \mbox{per $R2$} \\
\end{array}\]
Si osservi che l'espressione $1 + 0$ \`e in {\em forma normale} in questo sistema in quanto nessuna tra le regole $(R1)$-$(R4)$
si pu\`o applicare ad essa.

\medskip\noindent
Si dimostri che:
\begin{enumerate}
\item il termine $D_X(X * X)$ ha due diverse riduzioni, confluenti nella stessa forma normale;
\item se si aggiunge la regola 
		\[(R5) \qquad u + 0 \to u\] 
	esiste un termine che si riduce a due forme normali distinte;
\item si trovi una regola $(R6)$ da aggiungere alle regole $(R1)$-$(R5)$ tale che la confluenza del sistema venga ristabilita.
\end{enumerate}
}\end{exercise}


\begin{exercise}\label{ex:deduzione-naturale}
{\em
Sulla base dell'esempio:
\[
\prooftree
	\prooftree
		\prooftree
			\prooftree
				[A \to (B \to C)]^2
				\quad
				\prooftree
					[A \wedge B]^1
				\justifies
					A
				\using \wedge\, E_1
				\endprooftree
			\justifies
				B \to C
			\using \to E
			\endprooftree
			\quad
			\prooftree
				[A \wedge B]^1
			\justifies
				B
			\using \wedge\, E_2
			\endprooftree
		\justifies
			C
		\using \to E
		\endprooftree
	\justifies
		A \wedge B \to C
	\using \to I, -1
\endprooftree
\justifies
	(A \to (B \to C)) \to (A \wedge B \to C)
\using \to I, -2
\endprooftree
\]
si producano le dimostrazioni con la deduzione naturale delle seguenti formule:
\begin{enumerate}
\item $A \to (\neg \,A \to B)$
\item $(A \to B) \to ((B \to C) \to (A \to B))$
\item $(\neg\, B \to \neg\, A) \to (A \to B)$
\item $(A \wedge B \to C) \to (A \to (B \to C))$
\item $\forall x(A[x] \to B) \to (\exists x A[x] \to B)$ se $x \not \in \textit{FV}(B)$
\end{enumerate}
dove con $A[x]$ si intende una formula del calcolo dei predicati in cui $x$ possa occorrere libera.
}\end{exercise}

\begin{exercise}\label{ex:lambda-calcolo}
Si derivino i seguenti giudizi di tipo utilizzando le regole e l'assioma del $\lambda$-calcolo con tipi semplici:
\begin{enumerate}
\item $x : A, y : A \to B \vdash yx : B$
\item $\vdash \lambda x : A \to B\, \lambda y: B \to C\, \lambda z : A .\, y (xz): C$
\end{enumerate}
\end{exercise}

\section{Svolgimento esercizi}

\subsection{Svolgimento}

\paragraph{}

1) Si hanno due possibili riduzioni in base a dove si sceglie di applicare prima la regola $R1$.

\[\begin{array}{llll}
D_X(X * Y) & \to & (X * D_X(X)) + (D_X(X) + X) & \mbox{per $R4$} \\
                 & \to & (X * 1) + (D_X(X) + X) & \mbox{per $R1$} \\
                 & \to & (X * 1) + (1 + X) & \mbox{per $R1$} \\
\end{array}\]

\[\begin{array}{llll}
D_X(X * Y) & \to & (X * D_X(X)) + (D_X(X) + X) & \mbox{per $R4$} \\
                 & \to & (X * D_X(X)) + (1 + X) & \mbox{per $R1$} \\
                 & \to & (X * 1) + (1 + X) & \mbox{per $R1$} \\
\end{array}\]

\paragraph{}

2) Per esempio $D_X(X + 0)$ si può ridurre nei seguenti modi.

\[\begin{array}{llll}
D_X(X + 0) & \to & D_X(X) & \mbox{per $R5$} \\
           & \to & 1 & \mbox{per $R1$} \\
\end{array}\]

\[\begin{array}{llll}
D_X(X + 0) & \to & D_X(X) + D_X(0) & \mbox{per $R3$} \\
           & \to & 1 + D_X(0)& \mbox{per $R1$} \\
\end{array}\]

\paragraph{}

3) Per fare in modo che si abbia una sola forma normale è sufficiente aggiungere la seguente regola.

        \[(R6) \qquad D_X(0) \to 0\] 

così si ha:

\[\begin{array}{llll}
D_X(X + 0) & \to & D_X(X) + D_X(0) & \mbox{per $R3$} \\
           & \to & 1 + D_X(0)& \mbox{per $R1$} \\
           & \to & 1 + 0 & \mbox{per $R6$} \\
           & \to & 1 & \mbox{per $R5$} \\
\end{array}\]

\subsection{Svolgimento}

\paragraph{1) $A \to (\neg A \to B)$}

\[
\prooftree
	\prooftree
		\prooftree
			\prooftree
                        [\neg A]^2
                        \quad
                        [A]^1
			\justifies
				\perp
			\using \neg E
			\endprooftree
		\justifies
			B
		\using \perp E
		\endprooftree
	\justifies
		\neg A \to B
	\using \to I, -2
\endprooftree
\justifies
	A \to (\neg A \to B)
\using \to I, -1
\endprooftree
\]

\paragraph{2) $(A \to B) \to ((B \to C) \to (A \to B))$}

\[
		\prooftree
			\prooftree
				[A \to B]^1
	\justifies
		(B \to C) \to (A \to B)
	\using \to I
\endprooftree
\justifies
	(A \to B) \to ((B \to C) \to (A \to B))
\using \to I, -1
\endprooftree
\]

\paragraph{3) $(\neg\, B \to \neg\, A) \to (A \to B)$}

\[
\prooftree
	\prooftree
		\prooftree
			\prooftree
                        [A]^1
                        \quad
                            \prooftree
                            [\neg B \to \neg A]^2
                                    \quad
                                    [\neg B]^3
                                \justifies
                                    \neg A
                                \using \to E
                            \endprooftree
			\justifies
				\perp
			\using \neg E
			\endprooftree
		\justifies
			B
		\using RAA, -3
		\endprooftree
	\justifies
		A \to B
	\using \to I, -1
\endprooftree
\justifies
	(\neg B \to \neg A) \to (A \to B)
\using \to I, -1
\endprooftree
\]

\paragraph{4) $(A \wedge B \to C) \to (A \to (B \to C))$}

\[
\prooftree
	\prooftree
		\prooftree
                       [A \wedge B \to C]^1
			\justifies
				B \to C
			\using \wedge E_2
			\endprooftree
	\justifies
		A \to (B \to C)
	\using \to I
\endprooftree
\justifies
	(A \wedge B \to C) \to (A \to (B \to C))
\using \to I, -1
\endprooftree
\]

\paragraph{5) $\forall x(A[x] \to B) \to (\exists x A[x] \to B)$ se $x \not \in \textit{FV}(B)$}

\subsection{Svolgimento}

\paragraph{1) $x : A, y : A \to B \vdash yx : B$.}
\paragraph{$\Gamma$ = $x : A, y : A \to B$ }
\[
\prooftree
    \prooftree
    \justifies
    \Gamma \vdash y\: : A \to B
    \using ax
    \endprooftree
    \quad
    \prooftree
    \justifies
    \Gamma \vdash x\: : A
    \using ax
    \endprooftree
    \justifies
        \Gamma \vdash y\:x \: : B
        \using
        \to E
\endprooftree
\]

\paragraph{2) $\vdash \lambda x : A \to B\, \lambda y: B \to C\, \lambda z : A .\, y (x\:z):  (A \to B) \to (B \to C) \to A \to C$.}
\paragraph{$\Gamma$ = $x \: : \: A \to B$, $y \: : \: B \to C$, $z\: : \: A$, $y\:(x\:z)\: : \: C$}

\[
\prooftree
	\prooftree
 	\prooftree
  	\prooftree
   	\prooftree
\justifies
		\Gamma \vdash y\: :\: B \to C
\using ax
\endprooftree
\quad
	\prooftree
 \prooftree
\justifies
		\Gamma \vdash x\: : \: A \to B
\using ax
 \endprooftree
 \quad
 	\prooftree
\justifies
		\Gamma \vdash z\: : \: A
\using ax
\endprooftree
\justifies
		\Gamma \vdash x\:z \: : B
\using \to E
\endprooftree
\justifies
		\Gamma \vdash y(x\:z)\: : \: C
\using \to I
\endprooftree
\justifies
		\Gamma \vdash \lambda z \: : \: A . y(x\:z)\: : \: A \to C
\using \to I
\endprooftree
\justifies
		\Gamma \vdash \lambda y\: : \: B \to C\:\lambda z \: : \: A . y(x\:z)\: : \: (B \to C) \to A \to C
\using \to I
\endprooftree
\justifies
	\Gamma \vdash \lambda x\: :\: A \to B\:\lambda y\: : \: B \to C\:\lambda z \: : \: A . y(x\:z)\: : \: (A \to B) \to (B \to C) \to A \to C
\using \to I
\endprooftree
\]

\end{document}