\chapter{Terza consegna}

\section{Le sfaccettature dei programmi}

\paragraph{\textcolor{cyan}{I programmi come strumenti:}} 
questo punto di vista è, probabilmente, il più comune e il più intuitivo 
per la maggior parte delle persone. Nel percorso scolastico, infatti,
si vedono i programmi come strumenti per risolvere problemi, per esempio,
calcolare la media di un voto, oppure per svolgere un compito.

\paragraph{\textcolor{gray}{I programmi come opere dell'uomo:}}
una visione di questo tipo sorge spontanea quando si inizia a scrivere
codice e ci si trova davanti a un bivio in cui ci sono più possibili 
implementazioni di un algoritmo. Ci si rende conto che bisogna
fare delle scelte e che queste scelte possono presentare vantaggi e
svantaggi.

\paragraph{\textcolor{teal}{I programmi come oggetti fisici:}} 
durante l'ultimo anno del mio percorso scolastico della scuola
secondaria di secondo grado, ho potuto osservare alcuni aspetti più
fondazionali dell'informatica. Per esempio, dato che un programma ci
mette una determinata quantità di tempo e di energia per essere eseguito,
è possibile ottimizzarlo? Così mi sono imbattuto nei concetti di complessità
temporale e complessità spaziale. Inoltre ho anche potuto constatare 
che un programma nel "cloud" comunque verrà eseguito su un computer
fisico da qualche parte nel mondo. A livello universitario ci sono molti 
corsi che coprono questa sfaccettatura, come "Architettura degli elaboratori",
"Calcolabilità e complessità", "Sistemi operativi", etc...

\paragraph{\textcolor{violet}{I programmi come entità astratte:}}
questo aspetto non è emerso in modo esplicito come altri, ma si può 
facilmente intuire mediante concetti come "varaiabile" e "funzione"
che non sono comprensibili direttamente a livello fisico del calcolatore.
All'università invece vengono trattati in modo approfondito, in quanto
bagaglio fondamentale di ogni informatico.


\paragraph{\textcolor{orange}{I programmi come entità eseguibili:}}
il fatto di considerare un programma come un'entità eseguibile è
stato molto sottolineato dal mio professore alle superiori. Il linguaggio
che ha scelto di insegnarci è stato il C che si presta molto bene a
questo tipo di visione. Infatti, il C che integra caratteristiche di
basso livello e deve essere compilato bene perchè possa produrre dei file
eseguibili. 

\paragraph{\textcolor{red}{I programmi come manufatti linguistico-notazionali:}}
essendo uno studente del corso di "Linguaggi e Sistemi" posso guardare, in
retrospettiva, a quello che è stato insegnato alle superiori con occhio critico.
Per esempio mi è stata solo spiegata la sintassi del C ed è stata liquidata
con un "è così" senza spiegazioni approfondite. Un esempio è il confronto
(==) che, per la maggior parte del mio percorso scolastico, è stato "oscuro"
e arcaico. Quindi posso tranquillamente affermare che tra le sei sfaccettature
è stata quella più trascurata.


\section{Difficoltà nello studio dell'informatica}

La difficoltà che si incontra studiando informatica, come qualsiasi altra
disciplina, è altamente soggettivo, tuttavia credo che esistano degli \textit{scogli}
in cui gli studenti hanno più possibilità di inciampare. Per me, come si è già potuto
intuire, fu l'aspetto sintattico e linguistico. Pur non avendo alcuna difficoltà
nel risolvere problemi raramente riuscivo a prendere punteggio pieno per 
colpa di errori sintattici (come, per esempio = al posto di ==). 
Un'altra sfaccettatura che può, potenzialmente, creare problemi riguarda la 
visione dei programmi come entità astratte. Può essere difficile per gli
studenti concettualizzare qualcosa che non è tangibile. Si è visto, durante
questo corso \footnote{MTDI}, che spesso gli alunni fraintendono il significato
di "variabile" vedendola come una sorta di storico di tutti i valori che ha assunto
durante l'esecuzione. Oppure si ha difficoltà a concettualizzare il concetto
di "loop".
Per quanto riguarda la concezione di un programma come un oggetto fisico: raramente
viene affrontata in certi percorsi scolastici in cui ci si concentra più sullo
scrivere codice in linguaggi di alto livello in modo scollegato dal dispositivo 
fisico che lo esegue. Infine, il concetto di programma come opera dell'uomo
è molto difficile da affrontare in quanto richiede una certa maturità e una
certa esperienza oltre a una propensione a una visione più "umanistica" e "filosofica"
rispetto alle altre sfaccettature.

\section{Informatica alle superiori: modi di vedere un programma}

\paragraph{\textcolor{purple}{Primo anno:}} al primo anno, come prima consa
presenterei la visione come strumento, perchè è la più facile da comprendere
per chi si affaccia per la prima volta a questo mondo ed è quella con i risvolti
più pratici. Per esempio, il telefono che usi per divertirti o il computer che usi
per fare i compiti.

\paragraph{\textcolor{brown}{Secondo anno:}} al secondo anno ritengo opportuno
mostrare agli studenti che un programma può essere visto come entità astratta e
come entità eseguibile. Queste due sfaccettature sono molto legate tra loro e
si possono spiegare in modo congiunto. Esse mostrano qualcosa di non tangibile che,
può fare un po' paura all'inizio, ma proprio per il fatto che non è legato a una struttura
ben determinata è più interessante da studiare e mostra un numero quasi infinito di
possibilità. 

\paragraph{\textcolor{blue}{Terzo anno:}} a questo punto si potrebbe presentare,
finalmente, il calcolatore e il programma come oggetto fisico. In questo modo
si mostra agli allievi che tutto ciò che hanno studiato fin'ora ha anche dei risvolti
fisici e che non è solo un'astrazione.
 
\paragraph{\textcolor{lightgray}{Quarto anno:}} al quarto anno è tempo di
fare un'analisi della sintassi di un linguaggio e di come può essere scritto
un programma in modo che sia leggibile e comprensibile da altre persone.

\paragraph{\textcolor{olive}{Quinto anno:}} giunti alla fine del percorso
di studi si può iniziare a parlare di un argomento molto delicato, ossia
il programma come opera dell'uomo. Ciò prevede che gli studenti, negli anni
trascorsi abbiano raggiunto un'adeguata consapevolezza per poter discutere in
modo critico e costruttivo lo scopo con cui un programma è stato scritto e
il fatto che porta con sè le idee e i valori del suo creatore.
