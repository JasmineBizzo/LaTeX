\chapter{Quinta consegna}

\section{PRIMM}

\dfn{PRIMM}{
    PRIMM è un framework per la progettazione di attività didattiche
    per l'insegnamento della programmazione comprendente le seguenti fasi:
    \begin{itemize}
        \item \textbf{Predizione:} gli studenti devono prevedere il
        comportamento di un programma;
        \item \textbf{Esecuzione:} gli studenti devono eseguire il programma
        e verificare la predizione;
        \item \textbf{Investigazione:} gli studenti devono correggere la
        predizione in caso di errore;
        \item \textbf{Modifica:} gli studenti devono modificare il programma
        in modo che si comporti in un modo diverso;
        \item \textbf{Risoluzione:} gli studenti devono risolvere un problema
        usando il programma.
    \end{itemize}

}

\paragraph{\textcolor{blue}{Pro:}}

\begin{itemize}
    \item [$\Rightarrow$] ;
\end{itemize}

\paragraph{\textcolor{red}{Contro:}}

\begin{itemize}
    \item [$\Rightarrow$] ;
\end{itemize}

\section{POGIL}

\dfn{POGIL}{
    POGIL è un framework per la progettazione di attività didattiche
    per l'insegnamento della programmazione. Durante queste attività gli
    studenti attraversano un ciclo di esplorazione, concettualizzazione e
    applicazione.
    Gli studenti scoprono i concetti chiave e costruiscono la propria
    conoscenza attraverso l'interazione con i compagni e con il docente.
}

\paragraph{\textcolor{blue}{Pro:}}

\begin{itemize}
    \item [$\Rightarrow$] ;
\end{itemize}

\paragraph{\textcolor{red}{Contro:}}

\begin{itemize}
    \item [$\Rightarrow$] ;
\end{itemize}

\section{NLD}

\dfn{NDL}{
    NDL (Necessity design learning) è un framework per la progettazione di attività didattiche
    per l'insegnamento della programmazione. Sostanzialmente si dà un problema
    risolvibile con un certo costrutto senza introdurlo. Successivamente,
    prima che lo studente si scoraggi, si introduce il costrutto.
}

\paragraph{\textcolor{blue}{Pro:}}

\begin{itemize}
    \item [$\Rightarrow$] ;
\end{itemize}

\paragraph{\textcolor{red}{Contro:}}

\begin{itemize}
    \item [$\Rightarrow$] ;
\end{itemize}

\section{Riflessioni sugli approcci}