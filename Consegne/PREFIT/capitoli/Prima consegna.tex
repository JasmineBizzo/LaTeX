\chapter{Consegna 1}

\section{Punti su cui sono d'accordo}
\subsection{Frasi}
\begin{enumerate}
    \item "\textit{Certe attività cognitive non sono più dominio esclusivo dell’umanità: lo vediamo in tutta una serie di giochi da scacchiera (dama, scacchi, go, …) un tempo unità di misura per l’intelligenza e nei quali ormai il computer batte regolarmente i campioni del mondo.}", \textbf{Informatica: la terza rivoluzione "dei rapporti di potere"};  
    \item "\textit{Per preparare i cittadini alla società industriale, nei due secoli passati, non sono state date agli studenti competenze operative sui macchinari industriali, ma sono state inserite nelle scuole le discipline scientifiche che ne spiegavano i princìpi scientifici alla base.}", \textbf{Informatica e competenze digitali: cosa insegnare?};
    \item "\textit{Le tecnologie digitali dovrebbero essere progettate per promuovere la democrazia
e l‘inclusione. }", \textbf{Manifesto di Vienna per l'umanesimo digitale}.
\end{enumerate}

\subsection{Motivazioni}
\begin{enumerate}
    \item Attualmente l'essere umano con l'ELO\footnote{Sistema di valutazione della forza relativa di uno scacchista} più elevato è \textbf{Magnus Carlsen}\footnote{Campione del mondo dal 2013 al 2023} che ha raggiunto \href{https://www.chess.com/it/article/view/magnus-carlsen-campione-mondo-scacchi-retrospettiva#elo}{un picco di 2882 nella variante classica}. \textbf{Stockfish}, il più forte motore scacchistico attualmente esistente, ha un ELO stimato di circa 3600. Una differenza abisssale. Inoltre è molto interessante osservare le partite tra vari computers in cui la probabilità che Stockfish perda con il bianco è quasi e, inoltre riesce anche a vincere con il nero\footnote{Statisticamente, ad alti livelli, è molto più difficile vincere con il nero dato che non si ha il bonus dell'iniziativa}. Alcune delle mosse che vengono fatte in questi "scontri" sono basati su calcoli molto complessi e per molti esseri umani, inclusi dei granmaestri (\textbf{GM}), risultano innaturali. Oggi vengono usati \textit{engine}, dagli stessi GM, per allenarsi poichè un essere umano non giocherà sempre la mossa migliore, un computer sì;
    \item Come ricordato nell'articolo sono importanti sia le competenze digitali che quelle informatiche, tuttavia bisogna mettere in evidenza una realtà ineluttabile: le tecnologie si \textbf{evolvono continuamente}. Ciò significa che anche gli applicativi cambino nel tempo e, il compito della scuola non è solo quello di preparare i ragazzi al momento corrente, ma di fornire le capacità per \textbf{adattarsi} ai nuovi modelli. Ciò può essere fatto insegnando l'informatica perchè, anche se passa il tempo, le strutture fondamentali non ricevono quasi nessun cambiamento\footnote{Ci possono essere delle "rivoluzioni" estreme nel modo di concepire la base di una tecnologia, ma sono casi estremi e limitati}. Questo fa si che "insegnare la programmazione" sia molto più efficace che "insegnare un determinato linguaggio" o, per mantenerci più sul generico "insegnare come si usa un editor di testo" sia più efficacie che "insegnare come si usa Word/Office";
    \item Tutt@  dovrebbero essere liber@ di esprimersi liberamente nella società della tecnologia. Purtroppo l'aumento della platea dei beneficiari di queste tecnologie può aver contribuito ad aumentare l'odio verso il diverso dando voce a persone la cui opinione, un tempo, sarebbe stata relegata alle quattro mura di un bar il sabato sera. La frase in sè e per sè è corretta, ma per trasformare l'inchiostro in realtà ci vorrà ancora molto tempo e forse non si raggiungerà mai una tecnologia tale da permettere una profonda e sincera "\textbf{inclusione}".
\end{enumerate}

\section{Punti su cui non sono d'accordo}
\subsection{Frasi}
\begin{enumerate}
    \item
\end{enumerate}

\subsection{Motivazioni}
\begin{enumerate}
    \item 
\end{enumerate}

\section{Considerazioni sui 3 paradigmi}
\subsection{Analisi dei 3 paradigmi}
\begin{itemize}
    \item \textbf{Paradigma matematico:} questo paradigma trova la sua massima espressione nei linguaggi funzionali. Per esempio, in Haskell\footnote{Linguaggio funzionale puro} ogni cosa è una funzione matematica. Il punto di forza dei linguaggi funzionali (lazy) è che la loro correttezza è vera a priori per cui, se un programma viene eseguito allora è corretto. Inoltre esistono linguaggi come Agda, Coq, etc. che servono per dimostrare matematicamente la correttezza formale di alcuni programmi (purtroppo non sono Turing completi);
    \item \textbf{Paradigma ingegneristico:} si basa su un intenso uso di Unit test e testing generici per assicurare una "correttezza" su un ampio insieme di casi (spesso nei casi limite). Questo implica una correttezza a posteriori per cui, il programma deve prima essere eseguito;
    \item \textbf{Paradigma scientifico:} si effettuano delle ipotesi e delle deduzioni. Un esempio di ciò è la logica di Floyd-Hoare in cui vengono poste delle pre-condizioni (ipotesi sui dati) e delle post-condizioni (dati attesi se il programma è corretto). 
\end{itemize}

\subsection{Conclusioni generali}

Tutti e tre i paradigmi mostrano un differente modo di osservare la realtà. É difficile metterli in una classifica o dire quale sia il più corretto perchè ciò dipende in gran parte dal background personale del singolo: un logico probabilmente sarà orientato verso il paradigma matematico, un tecnico verso quello ingegneristico e un chimico o un biologo verso quello scientifico. L'informatica è qualcosa di troppo complesso per essere ridotto a un solo di questi paradigmi. Come mostrato nella sezione precedente ogni paradigma include un pezzo dell'informatica quindi è inutile affermare che appartiene al paradigma X o al paradigma Y\footnote{Per usare un termine informatico possiamo dire che l'informatica, così come C++ o Java, è \textbf{multiparadigma}}

\section{L'informatica può essere considerata una scienza?}

\subsection{I requisiti}

\begin{itemize}
    \item Organizzati per comprendere, sfruttare e far fronte a un fenomeno pervasivo: esiste un intero campo di applicazione dell'informatica a questo livello ossia la \textbf{data science};
    \item Comprende i processi naturali e artificiali del fenomeno: l'informatica vuole comprendere e replicare in modo automatico dei fenomeni sia naturali che artificiali. Basti vedere alcune strutture chiaramente ispirate alla natura come gli \textbf{alberi}; 
    \item
\end{itemize}