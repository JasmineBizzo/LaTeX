\chapter{Domande e risposte}

\section{La natura dell'informatica}

\qs{}{Quali sono i problemi dell'informatica?}

\paragraph{Risposta:} 

\qs{}{Quali sono i problemi relativi all'insegnamento dell'informatica nel nostro sistema scolastico?}

\paragraph{Risposta:} 

\qs{}{Perchè è importante insegnare informatica fin dalla scuola primaria?}

\paragraph{Risposta:} 

\qs{}{Perchè è importante riflettere sulla natura dell'informatica prima di insegnarla?}

\paragraph{Risposta:} 

\qs{}{Quali sono le 3 anime/paradigmi che abbiamo discusso per inquadrare la natura dell'informatica?}

\paragraph{Risposta:} 

\section{Teorie dell'apprendimento}

\qs{}{Che cos'è il comportamentismo?}

\paragraph{Risposta:}

\qs{}{Che cos'è il cognitivismo?}

\paragraph{Risposta:}

\qs{}{Che cos'è il costruttivismo? (socio-costruttivismo e costruttivismo cognitivo)}

\paragraph{Risposta:}

\qs{}{Che cos'è il costruzionismo?}

\paragraph{Risposta:}

\qs{}{Quali sono i punti fondamentali del brano di Papert "Gli ingranaggi della mia infanzia"?}

\qs{}{Cos'è l'assimilazione?}

\paragraph{Risposta:} l'assimilazione è l'incorporazione di un determinato concetto in uno schema che è stato già acquisito.

\qs{}{Cos'è l'accomodamento?}

\paragraph{Risposta:} l'accomodamento è la modifica di una struttura cognitiva in relazione al contatto con nuove informazioni.

\qs{}{Cos'è la zona di sviluppo prossimo (ZSP)?}

\paragraph{Risposta:} la zona di sviluppo prossimo è la seconda delle tre aree di apprendimento di un bambino. Nella ZSP il bambino è in grado di apprendere solo con il supporto del docente ed è in quest'area che l'insegnante può intervenire.

\qs{}{Quali sono le caratteristiche principali dell'apprendimento attivo?}

\paragraph{Risposta:}