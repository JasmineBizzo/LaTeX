\chapter{Domande}

\section{Lullo, Leibniz e la storia del calcolo}

\qs{}{Che cos'è l'Ars Magna di Lullo?}

\paragraph{Risposta:} L'Ars Magna di Lullo è un'opera scritta da Raimondo Lullo, un filosofo, teologo e mistico catalano,
che trattava di un metodo per risolvere i problemi logici attraverso l'uso di diagrammi e figure.
In questo modo si potevano raggiungere verità in ogni campo del sapere.

\subsubsection{}

\qs{}{Quali sono le funzioni della combinatoria di Lullo?}

\paragraph{Risposta:} L'Ars Combinatoria di Lullo è un metodo inventivo
che permette di elaborare dimostrazioni allo scopo di convertire gli "infedeli" al cristianesimo.   

\subsubsection{}

\qs{}{Elencare almeno tre dei principi alla base della caratteristica universale di Leibniz.}

\paragraph{Risposta:}

\begin{itemize}
    \item [$\Rightarrow$] Le idee sono analizzabili;
    \item [$\Rightarrow$] L'analisi termina con le idee primitive;
    \item [$\Rightarrow$] Le idee possono essere rappresentate da simboli;
    \item [$\Rightarrow$] Le relazioni tra le idee possono essere rappresentate da simboli;
    \item [$\Rightarrow$] Le idee possono essere combinate per ottenere nuove idee tramite opportune regole.
\end{itemize}

\subsubsection{}

\qs{}{Elencare le principali caratteristiche della lingua universale.}

\paragraph{Risposta:}

\begin{itemize}
    \item [$\Rightarrow$] Segni che rappresentano direttamente le nozioni e le cose, non le parole;
    \item [$\Rightarrow$] Segni composti di figure geometiche e di pitture;
    \item [$\Rightarrow$] Direttamente collegata con l'enciclopedia;
    \item [$\Rightarrow$] Connessioni tra i caratteri corrispondono alle connessioni tra le cose;
    \item [$\Rightarrow$] I caratteri della lingua universale esprimono relazioni tra pensieri.
\end{itemize}

\subsubsection{}

\qs{}{Fare un esempio di proposizione dal frammento XX di Leibniz.}

\paragraph{Risposta:} Se $A \geq B$ e $B \geq C$, allora $A \geq C$.

\subsubsection{}

\qs{}{Qual è l'idea alla base di un approccio "pointless" alle nozioni geometriche e topologiche?}

\paragraph{Risposta:} L'idea alla base di un approccio "pointless" è quella che la classe
degli spazi aperti di uno spazio topologico costituisce un reticolo rispetto all'inclusione
per cui vale la proprietà di distributività.

\subsubsection{}

\qs{}{Che cosa è lo Entscheidungsproblem formulato da Hilbert e Ackermann nel 1928?}

\paragraph{Risposta:} L'Entscheidungsproblem è un problema matematico che consiste nel determinare
un modo combinatorio finito per il quale combinazioni di simboli primitivi conducono a dimostrazioni.
In altre parole, trovare una procedura che consente di decidere la validità di una data
espressione logica con un numero finito di operazioni.

\section{Turing e la fisica del calcolo}

\qs{}{Perché Turing richiede che i simboli che possono essere scritti sul nastro 
di una macchina di Turing siano elementi di un insieme finito?}

\paragraph{Risposta:} Perché se l'insieme dei simboli fosse infinito, ci sarebbero un numero infinito di 
simboli molto simili tra loro, rendendo impossibile distinguerli.

\subsubsection{}

\qs{}{Perché Turing richiede che gli stati che possono essere assunti dall'unità
 di controllo di una macchina di Turing siano elementi di un insieme finito?}

\paragraph{Risposta:} Perché c'è un limite alla percezione delle caselle del nastro determinato dallo stato mentale.

\subsubsection{}

\qs{}{Che cosa stabilisce il procedimento diagonale di Cantor?}

\paragraph{Risposta:} Il procedimento diagonale di Cantor stabilisce che non tutte le funzioni 
sono calcolabili da una macchina di Turing.

\subsubsection{}

\qs{}{Qual è la relazione tra una macchina di Turing generale ed un automa finito?}

\paragraph{Risposta:} Un automa finito è una macchina di Turing generale con un nastro finito.

\subsubsection{}

\qs{}{Quali sono state le tappe fondamentali della formalizzazione degli automi finiti?}

\paragraph{Risposta:}

\begin{itemize}
    \item [$\Rightarrow$] Reti di neuroni di McCulloch e Pitts;
    \item [$\Rightarrow$] Eventi regolari di Kleene;
    \item [$\Rightarrow$] Lavoro di Rabin e Scott.
\end{itemize}

\subsubsection{}

\qs{}{Che cos'è il gioco della vita di Conway?}

\paragraph{Risposta:} Il gioco della vita di Conway è un automa cellulare bidimensionale (gioco a zero giocatori).
Il risultato di ogni generazione è determinato dallo stato della precedente secondo semplici regole.

\subsubsection{}

\qs{}{Qual è la relazione tra termodinamica e teoria dell'informazione?}

\paragraph{Risposta:} Von Neumann cercò di valutare il costo energetico minimo 
per un atto elementare di generazione di informazione ($kT \ln N$).

\subsubsection{}

\qs{}{Che cos'è il Principio di Landauer?}

\paragraph{Risposta:} Le operazioni logicamente irreversibili (cancellazione di bit) 
generano entropia pari alla quantità di informazione cancellata.

\subsubsection{}

\qs{}{Quali sono le difficoltà nell'applicazione della logica matematica classica alla formalizzazione di automi 
che descrivano sia i computer che il cervello umano?}

\paragraph{Risposta:} L'assiomatizzazione a scatole nere non è sufficiente per descrivere
il comportamento di un computer in relazione al comportamento del cervello umano.

\subsubsection{}

\qs{}{Quali sono, secondo Landauer, le operazioni che aumentano l'entropia nel processo di calcolo?}

\paragraph{Risposta:} Le operazioni che aumentano l'entropia sono quelle che comportano la cancellazione di bit.

\section{Funzioni calcolabili e combinatori}

\qs{}{Quali sono gli ingredienti principali di un sistema formale nella formulazione di Curry?}

\paragraph{Risposta:} I combinatori, la trattazione di funzioni a più argomenti
come funzioni a un solo argomento e la sostituzione.

\subsubsection{}

\qs{}{Quando una regola è ammissibile in un sistema formale?}

\paragraph{Risposta:} Una regola è ammissibile in un sistema formale se è possibile
dimostrare che essa è derivabile dalle regole di base del sistema.

\subsubsection{}

\qs{}{Indicare almeno due applicazioni dei sistemi formali alla formalizzazione di processi di riscrittura.}

\paragraph{Risposta:}

\begin{itemize}
    \item [$\Rightarrow$] Combinatori;
    \item [$\Rightarrow$] Lambda calcolo.
\end{itemize}

\subsubsection{}

\qs{}{Come si può caratterizzare la ricorsione strutturale?}

\paragraph{Risposta:} Con una teoria generale della sostituzione.

\subsubsection{}

\qs{}{Come può essere utilizzata la ricorsione strutturale in un linguaggio di programmazione?}

\paragraph{Risposta:} La ricorsione strutturale può essere utilizzata per definire funzioni ricorsive e tipi induttivi\footnote{Visto in "Metodi Formali dell'informatica".}.

\subsubsection{}

\qs{}{Quando è opportuno usare la coricorsione per la definizione di funzioni?}

\paragraph{Risposta:} Per esempio con iteratori o generatori.

\subsubsection{}

\qs{}{Che cosa è lo schema di ricorsione primitiva?}

\paragraph{Risposta:} Lo schema di ricorsione primitiva è un metodo per definire funzioni di numeri 
naturali $f(x, n)$ a partire da funzioni predefinite $g(x)$ e $h(x, n, f(x, n))$ mediante lo schema:

$$
\begin{cases}
    f(x, 0) = g(x) \\
    f(x, n+1) = h(x, n, f(x, n))
\end{cases}
$$

\subsubsection{}

\qs{}{Che cosa è l'operatore di ricerca non limitato utilizzato da Kleene per definire le funzioni ricorsive generali?}

\paragraph{Risposta:} Dato un predicato $P(x, y)$, viene restituito il più piccolo $y$ tale che $P(x, y)$:

$$
f(x) = \mu y P(x, y)
$$

\subsubsection{}
\pagebreak
\qs{}{Dare un esempio di almeno due combinatori con le relative uguaglianze caratteristiche.}

\paragraph{Risposta:}

\begin{itemize}
    \item [$\Rightarrow$] $K x y = x$;
    \item [$\Rightarrow$] $Y f = f (Y f)$.
    \item [$\Rightarrow$] $I x = x$;
    \item [$\Rightarrow$] $B x y z = x (y z)$;
    \item [$\Rightarrow$] $C x y z = x z y$;
    \item [$\Rightarrow$] $S x y z = x z (y z)$.
\end{itemize}

\subsubsection{}

\qs{}{Perché Curry chiamava il combinatore Y il "combinatore paradossale"?}

\paragraph{Risposta:} Perché il combinatore Y è un combinatore che si autoapplica.

\subsubsection{}

\qs{}{Come si può motivare il tipo a $\rightarrow$ (b $\rightarrow$ a) per il combinatore K? }

\paragraph{Risposta:} Prendiamo un qualsiasi x di tipo a e un qualsiasi y di tipo b: poiché Kxy = x, il tipo di Kxy deve essere lo stesso del tipo di x, cioè a. Poichè Kx prende come argomento un b e restituisce un a, il suo tipo è b $\rightarrow$ a. Poiché K prende come argomento un a e restituisce un b $\rightarrow$ a, ha tipo a $\rightarrow$ (b $\rightarrow$ a).

\subsubsection{}

\qs{}{Che cosa si intende con "isomorfismo di Curry-Howard"?}

\paragraph{Risposta:} L'isomorfismo di Curry-Howard è una corrispondenza tra dimostrazioni logiche
e programmi informatici.

\section{"As we may think"}

\subsection{Sezione 1}

\qs{}{Formulare sinteticamente il complesso delle problematiche relative all'organizzazione dell'informazione.}

\paragraph{Risposta:} Si ha difficoltà nel trovare le informazioni che si stanno cercando.
Il problema deriva dal fatto che si utilizza l'indicizzazione (per nome, autore, etc.), ma 
la mente umana funziona per associazione, collegando idee tra loro.

\subsection{Sezione 6}

\qs{}{Isolare quattro problemi fondamentali delle classificazioni tradizionali (e dei relativi processi di indicizzazione).}

\paragraph{Risposta:}

\begin{itemize}
    \item [$\Rightarrow$] Artificiosità dei sistemi di indicizzazione;
    \item [$\Rightarrow$] L'informazione deve essere cercata da sottoclasse a sottoclasse;
    \item [$\Rightarrow$] Bisogna avere delle regole per localizzare le informazioni;
    \item [$\Rightarrow$] Dopo aver trovato l'elemento cercato bisogna tornare indietro.
\end{itemize}

\subsubsection{}

\qs{}{Caratterizzare l'operazione di associazione secondo Bush.}

\paragraph{Risposta:} 

\begin{itemize}
    \item [$\Rightarrow$] La mente umana scatta istantaneamente da un'idea all'altra;
    \item [$\Rightarrow$] Gli elementi non sono fissi, ma si modificano;
    \item [$\Rightarrow$] La memoria umana è associativa, ma transitoria.
\end{itemize}

\subsection{Sezione 7}

\qs{}{Elencare in modo analitico le operazioni compiute nel collegare due elementi mediante il memex. Qual è la funzione del codice e del libro dei codici?}

\paragraph{Risposta:} Un utente che crea un percorsolo nomina, lo inserisce nel libro dei codici e 
lo batte sulla sua tastiera. L'utente preme un singolo tasto e il memex collega i due elementi.
Da quel momento in poi, ogni volta che l'utente preme un elemento del percorso, il memex richiama
l'elemento collegato.

\subsubsection{}

\qs{}{Discutere la seguente affermazione: la creazione di percorsi associativi contraddice la struttura lineare del testo scritto.}

\paragraph{Risposta:} Il testo scritto è, per sua natura, immutabile, fissato in un ordine lineare. Mentre l'associazione è 
un processo dinamico, che si evolve nel tempo. Si possono creare percorsi associativi, ma il testo scritto rimane invariato.

\subsubsection{}

\pagebreak

\qs{}{Descrivere una struttura di dati che permetta una implementazione (astratta) delle piste (trails).}

\paragraph{Risposta:} Una struttura dati che permetta di implementare le piste è un grafo. In un grafo
ogni nodo rappresenta un'idea e ogni arco rappresenta un collegamento tra due idee\footnote{Per esempio, su Obsidian (tool per MarkDown), esistono dei grafi che mostrano i collegamenti tra i propri appunti.}.

\subsubsection{}

\qs{}{Quali sono le principali operazioni di organizzazione della conoscenza compiute dall'utilizzatore del memex nell'esempio dell'arco?}

\paragraph{Risposta:} L'utente del memex, nell'esempio dell'arco:

\begin{itemize}
    \item [$\Rightarrow$] Apre un'enciclopedia, trova un articolo e lo lascia proiettato;
    \item [$\Rightarrow$] Apre un libro di storia, trova un articolo pertinente;
    \item [$\Rightarrow$] Collega i due articoli;
    \item [$\Rightarrow$] Continua a creare collegamenti;
    \item [$\Rightarrow$] Occasionalmente lascia dei commenti e aggiunge note scritte a mano da lui;
    \item [$\Rightarrow$] Diversi anni dopo e in grado di mostrare a un suo amico i percorsi che ha creato;
    \item [$\Rightarrow$] Trasferisce i percorsi sul Memex del suo amico in modo che egli possa continuare il lavoro.
\end{itemize}

\subsubsection{}

\qs{}{In quale senso le enciclopedie generate con il memex sono di un tipo totalmente nuovo?}

\paragraph{Risposta:} Sono personalizzabili e non sono statiche, ma dinamiche.

\subsubsection{}

\qs{}{In che cosa consiste la nuova professione di "apripista" (trail blazer)\footnote{Honkai Star Rail reference}?}

\paragraph{Risposta:} Gli apripista sono persone che esplorano nuovi collegamenti tra idee.
\pagebreak
\section{Engelbart e Nelson}

\qs{}{Quali sono le caratteristiche dei problemi che hanno motivato il lavoro di Engelbart sul progetto di aumentazione dell'intelligenza umana?} 

\paragraph{Risposta:} I problemi wicked.

\begin{itemize}
    \item [$\Rightarrow$] I problemi sono mal definiti;
    \item [$\Rightarrow$] Non c'è un punto in cui si possa dire che il problema è risolto;
    \item [$\Rightarrow$] Le soluzioni sono soggettive (non giuste o sbagliate);
    \item [$\Rightarrow$] Non si può procedere per tentativi;
    \item [$\Rightarrow$] Ogni problema è un caso a sé;
    \item [$\Rightarrow$] Non c'è una soluzione alterntiva preassegnata.
\end{itemize}

\subsubsection{}

\qs{}{Fare almeno due esempi di problema wicked e due esempi di problema tame.}

\paragraph{Problemi wicked:}

\begin{itemize}
    \item [$\Rightarrow$] Problemi urbanistici;
    \item [$\Rightarrow$] Organizzare una mostra.
\end{itemize}

\paragraph{Problemi tame:}

\begin{itemize}
    \item [$\Rightarrow$] Risolvere un'equazione;
    \item [$\Rightarrow$] Racimolare 10.000 euro.
\end{itemize}

\subsubsection{}

\qs{}{Che cosa intende Engelbart con "problema complesso"?}

\paragraph{Risposta:} Un problema complesso è un problema che non può essere affrontato ricorrendo a trucchetti, ma deve essere compreso.

\subsubsection{}

\qs{}{Che cosa intende Engelbart con "incremento delle capacità" (increased capability)?}

\paragraph{Risposta:} L'incremento delle capacità intellettuali umane che porti alla simbiosi
tra uomo e macchina.

\subsubsection{}

\qs{}{Che cosa intende Engelbart con "sistema H-LAM/T"? Spiegare sinteticamente ciascuno dei termini coinvolti nell'acronimo.}

\paragraph{Risposta:} Il sistema H-LAM/T\footnote{I termini sono spiegati nella sezione \ref{augmentation}.} è un sistema che permette di aumentare le capacità umane.

\begin{itemize}
    \item [$\Rightarrow$] H: Human;
    \item [$\Rightarrow$] L: Language;
    \item [$\Rightarrow$] A: Artefacts;
    \item [$\Rightarrow$] M: Methodology;
    \item [$\Rightarrow$] T: Training.
\end{itemize}








\subsection{Engelbart}

\subsection{Nelson}

\section{Otlet, Lickider e Kay}

\subsection{Otlet}

\subsection{Licklider}

\subsection{Alan Kay}