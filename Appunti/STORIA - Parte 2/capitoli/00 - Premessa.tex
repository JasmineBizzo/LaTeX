\chapter{Premessa}

\section{Licenza}

Questi appunti sono rilasciati sotto licenza Creative Commons Attribuzione 4.0 Internazionale (per maggiori
informazioni consultare il link: \href{https://creativecommons.org/version4/}{https://creativecommons.org/version4/}). Sono basati sulle slides del corso "Storia dell'Informatica" del prof. Felice Cardone.
\begin{center}
    \includegraphics[width=0.5\textwidth]{images/cc.png}
\end{center}

\section{Formato utilizzato}

In questi appunti vengono utilizzati molti \fancyglitter{box}. Questa è una semplice 
rassegna che ne spiega l'utilizzo:

\subsubsection{Box di "Corollario":}

\cor{Nome del corollario}{
    Testo del corollario. Per corollario si intende una definizione minore,
    legata a un'altra definizione.
}

\subsubsection{Box di "Definizione":}

\dfn{Nome delle definizione}{
    Testo della definizione.
}

\subsubsection{Box di "Domanda":}

\qs{}{
    Testo della domanda. Le domande sono spesso utilizzate per far riflettere
    sulle definizioni o sui concetti.
}

\subsubsection{Box di "Esempio":}

\ex{Nome dell'esempio}{
    Testo dell'esempio. Gli esempi sono tratti dalle slides del corso.
}

\subsubsection{Box di "Note":}

\nt{
    Testo della nota. Le note sono spesso utilizzate per chiarire concetti
    o per dare informazioni aggiuntive.
}

\subsubsection{Box di "Meta-nota":}

\clm{}{}{Testo della meta-nota. Le meta-note utilizzate 
    com note più personali per fornire riflessioni e motivazioni al 
    perchè si sta facendo ciò che si sta facendo. Vengono usate per 
    simulare lo stile espositivo del prof. Cardone che spazia in poco tempo 
    su molti argomenti.
    Possono essere scollegate direttamente dal contesto in cui si trovano e ai fini degli appunti
    possono essere tranquillamente ignorate.
}

\subsubsection{Box di "Teorema":}

\thm{Nome del teorema}{
    Testo del teorema. 
}