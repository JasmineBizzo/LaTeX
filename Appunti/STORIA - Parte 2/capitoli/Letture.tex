\chapter{Letture}

Questo capitolo contiene dei mini-riassunti delle Letture
obbligatorie per l'esame. Poiché è possibile e consigliato consultare
i documenti durante l'esame, questi riassunti hanno la funzione di 
aiutare a ricordare i concetti principali per facilitare la navigazione 
e la consultazione.

\begin{itemize}
    \item [$\Rightarrow$] Vannevar Bush, \href{https://informatica.i-learn.unito.it/pluginfile.php/368484/mod_resource/content/7/Bush%20%281945%29.pdf}{Come potremmo pensare}, 1945;
    \item [$\Rightarrow$] Susan Barnes, \href{https://informatica.i-learn.unito.it/pluginfile.php/368474/mod_resource/content/2/Barnes%2C%20S.B.%20--%20Douglas%20Carl%20Engelbart-%20developing%20the%20underlying%20concepts%20for%20contemporary%20computing.pdf}{Douglas Carl Engelbart: Developing the Underlying Concepts for Contemporary Computing}, 1997;
    \item [$\Rightarrow$] J. C. R. Licklider, \href{https://informatica.i-learn.unito.it/pluginfile.php/368452/mod_resource/content/2/Licklider%20-%20Man-Computer%20Symbiosis.pdf}{Man-computer symbiosis}, 1960;
    \item [$\Rightarrow$] Lev Manovich, \href{https://informatica.i-learn.unito.it/pluginfile.php/368438/mod_resource/content/1/Lev%20Manovich-Software_Takes_Command-Ch1.pdf}{Alan Kay's Universal Media Machine}, 2013.
\end{itemize}

\section{Come potremmo pensare}

\section{Douglas Carl Engelbart: Developing the Underlying Concepts for Contemporary Computing}

\section{Man-computer symbiosis}

\section{Alan Kay's Universal Media Machine}