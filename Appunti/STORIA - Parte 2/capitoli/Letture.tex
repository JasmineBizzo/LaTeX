\chapter{Letture}

Questo capitolo contiene dei mini-riassunti delle Letture
obbligatorie per l'esame. Poiché è possibile e consigliato consultare
i documenti durante l'esame, questi riassunti hanno la funzione di 
aiutare a ricordare i concetti principali per facilitare la navigazione 
e la consultazione.

\begin{itemize}
    \item [$\Rightarrow$] Vannevar Bush, \href{https://informatica.i-learn.unito.it/pluginfile.php/368484/mod_resource/content/7/Bush%20%281945%29.pdf}{Come potremmo pensare}, 1945;
    \item [$\Rightarrow$] Susan Barnes, \href{https://informatica.i-learn.unito.it/pluginfile.php/368474/mod_resource/content/2/Barnes%2C%20S.B.%20--%20Douglas%20Carl%20Engelbart-%20developing%20the%20underlying%20concepts%20for%20contemporary%20computing.pdf}{Douglas Carl Engelbart: Developing the Underlying Concepts for Contemporary Computing}, 1997;
    \item [$\Rightarrow$] J. C. R. Licklider, \href{https://informatica.i-learn.unito.it/pluginfile.php/368452/mod_resource/content/2/Licklider%20-%20Man-Computer%20Symbiosis.pdf}{Man-computer symbiosis}, 1960;
    \item [$\Rightarrow$] Lev Manovich, \href{https://informatica.i-learn.unito.it/pluginfile.php/368438/mod_resource/content/1/Lev%20Manovich-Software_Takes_Command-Ch1.pdf}{Alan Kay's Universal Media Machine}, 2013.
\end{itemize}

\section{Come potremmo pensare}

L'articolo si apre con una riflessione sulla seconda guerra mondiale (infatti l'articolo è del '45)
in cui vari scienziati (in particolare fisici) hanno contribuito allo sforzo bellico invece che alla ricerca.
Subito dopo si passa a parlare del fenomeno dell'accumulo di conoscenza (\fancyglitter{Information Overload}) e
di come il problema non fosse l'eccesso di informazione, ma la mancanza di strumenti per gestirla\footnote{
    Come esempio vengono viste le leggi della genetica di Mendel, passate inosservate per una generazione.
}.
Successivamente si parla della question costi-benefici relativa all'utilizzo di determinate tecnologie\footnote{
    Nè la macchina per i calcoli di Leibniz, nè la macchina analitica di Babbage sono state realizzate.
}.

Nella sezione 3 è presente una descrizione di un sistema per convertire la voce in testo, che permetterebbe
di velocizzare la scrittura di documenti e lascerebbe più tempo per la riflessione. Si parla di \fancyglitter{meccanizzazione
di processi ripetitivi}\footnote{Collegamento a Licklider, "Man-computer symbiosis"} e di come, in futuro,
si potrebbero avere macchine in grado di fare ragionamenti complessi a velocità molto elevate\footnote{
    Tali macchine saranno controllate mediante schede o pellicole (i microfilm adorati da Bush).
}.

Nella sezione 4 si parla di come il matematico non sia una persona che fa calcoli, ma una persona che risolve problemi
logici complessi (ad alto livello) e che per questo dovrebbe delegare i calcoli, anche di complessità elevata, a macchine. 
In futuro potranno esistere macchine sufficientemente meticolose da poter soddisfare
anche i matematici più esigenti.

Nella sezione 5 si parla di macchine in grado di fare ragionamenti logici prendendo in input delle premesse e restituendo
delle conclusioni. Per fare ciò nascerà un nuovo simbolismo, probabilmente posizionale. Inoltre non ci si limiterà a manipolare la logica,
ma anche le \fancyglitter{idee}. Dopo di ché, Bush, ritorna al problema della \fancyglitter{selezione} con l'esempio di 
un impiegato che deve trovare tutti gli impiegati che vivono a Trenton e conoscono lo spagnolo (lo fa tramite una macchina selezionatrice\footnote{
    Ricorda una SELECT in SQL.
}).
Però questa è una selezione semplice. Un altro esempio è quello di una centralina telefonica che deve connettersi a un'altra 
centralina, quindi senza ispezionare tutte le linee, ma solo quella che interessa. Termina
la sezione con ulteriori esempi e cita le \fancyglitter{biblioteche}.

Finalmente, nella sezione 6, si parla del problema dell'\fancyglitter{indicizzazione}. Il problema è che la mente umana 
non funziona per indicizzazione, ma per \fancyglitter{associazione}, facendo continui collegamenti tra idee\footnote{Ipertesti.}.
Difficilmente ciò può essere meccanizzato perfettamente al 100\%, ma si può, in parte, emulare. Questa 
è l'idea alla base del \fancyglitter{Memex}\footnote{Anche qua Bush si lascia trascinare dall'amore per i microfilm.}.
Bush parla di come sia importante il processo di collegare due elementi insieme. Nella 
settima sezione si fa un esempio del Memex che si vedrà nel famoso filmato.

Sezione 8: Bush parla di nuovi tipi di enciclopedie fruibili da tutti e personalizzate sulla base 
delle esigenze dell'utente. Inoltre viene introdotta la professione di \fancyglitter{apripista} (o trail blazers\footnote{Honkai Star Rail refernce.}):
persone che si occupano di creare nuovi collegamenti tra idee. Le ultime pagine sono dedicate a delle riflessioni
di natura più filosofica che consiglio di leggere.

\section{Douglas Carl Engelbart: Developing the Underlying Concepts for Contemporary Computing}

\paragraph{Introduzione:}

\paragraph{Influenze della formazione:}

\paragraph{Il concetto iniziale di Augmentation:}

\paragraph{Manipolazione simbolica:}

\paragraph{Computazione interattiva:}

\paragraph{Supporto da ARPA:}

\paragraph{THe Augmentation Research Center:}

\paragraph{The Mother of All Demos (Demonstration):}

\paragraph{Su ARPANET:}

\section{Man-computer symbiosis}

\section{Alan Kay's Universal Media Machine}