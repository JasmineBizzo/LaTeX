\chapter{Il \texorpdfstring{$\mathcal{\lambda}$}--calcolo non tipato}

\nt{Questo capitolo è concettualmente affine al capitolo "Il $\lambda$-calcolo" negli appunti del corso di "Linguaggi e paradigmi di programmazione"}

\section{Introduzione}

Il $\lambda$-calcolo fu introdotto nel 1933 da Alonzo Church. Con questo calcolo, Church, cercò di formalizzare la nozione di funzione calcolabile.

\nt{Non tutte le funzioni sono calcolabili. Alcuni dei motivi per cui è vero ciò sono spiegati nel corso di "Calcolabilità e complessità"}

\dfn{}{Sia Var = $\{x, \:y,\:z,\:...\}$ un insieme finito di variabili, la sintassi è la seguente:
$$M, N\:::= x \:\:\:|\:\:\: (\lambda x.M)\:\:\: |\:\:\: (M N)$$}

\nt{$\lambda x.M$ è un'astrazione o funzione con parametro formale x e corpo M}
\nt{$(M N)$ è l'applicazione delle funzione $M$ al parametro attuale $N$}

\section{Semantica}

\nt{Applicare una funzione $\lambda x.M$ a un argomento N significa valutare il corpo della funzione (M) in cui ogni occorrenza libera dell'argomento (x) è stata sostituita da N}

\dfn{Insieme delle variabili libere}{L'insieme delle variabili libere di un termine M, denotato come $fv(M)$, è definito induttivamente sulla struttura di M come segue:
$$fv(x) = \{x\} \:\:\:\: fv(\lambda x.M) = fv(M) \ /\{x\}\:\:\:\: fv(M N) = fv(M) \cup fv(N)$$
}

\dfn{Sostituzione}{
\begin{itemize}
    \item $x\{N/y\} = \begin{cases}
        N \text{ se } x = y \\
        x\; \text{ se } x \not = y
    \end{cases}$
    \item $(M_1 M_2) \{N/y\} = M_1\{N/y\} M_2\{N/y\}$;
    \item $(\lambda x.M)\{N/y\} = \begin{cases}
        \lambda x.M \:\:\:\:\:\:\:\:\:\:\:\:\:\:\:\:\:\:\:\:\:\:\:\:\text{ se } x = y \\
        \lambda x.M\{N/y\} \:\:\:\:\:\:\:\:\:\:\:\text{ se } x \not = y \text{ e } x \not\in fv(N) \\
        \lambda z.M\{z/x\}\{N/y\} \text{ se } x \not= y \text{ e } x \in fv(N) \text{ e } z \in Var - (fv(M) \cup fv(N))
    \end{cases}$
\end{itemize}
}

\dfn{$\alpha$-equivalenza}{L'$\alpha$-equivalenza  $\ac$ è la congruenza tra $\lambda$-espressioni tale che, se $y \not\in fv(M)$, allora $\lambda x.M \ac \lambda y.M\{y/x\}$}

\nt{$y \not\in fv(M)$ serve a evitare che una variabile libera in M venga catturata dalla congruenza}

\dfn{$\beta$-riduzione}{ La $\beta$-riduzione è la relazione tra $\lambda$-espressioni tale che:
\begin{itemize}
    \item $(\lambda x.M) N \br M\{N/y\}$;
    \item se $M\br M'$ allora $M N \br M' N$;
    \item se $M\br M'$ allora $M N \br N M'$;
    \item se $M\br M'$ allora $M N \br \lambda x.M'$.
\end{itemize}
}

\nt{Nella $\beta$-riduzione:

    $$(\lambda x.M) N \br M\{N/y\}$$

$(\lambda x.M) N$ è un \textit{$\beta$-redex}\footnote{REDucible EXpression}.

$M\{N/y\}$ è il suo \textit{ridotto}.

Ci possono essere più modi di ridurre la stessa $\lambda$-espressione. La riduzione di un $\beta$-redex può creare altri $\beta$-redex. La riduzione di un $\beta$-redex può cancellare altri $\beta$-redex. La riduzione può non terminare.}

\dfn{Church-Rosser nel $\lambda$-calcolo}{$R$ è confluente o Church-Rosser (CR) se
$$\forall M,\:\:N,\:\:L. \:\:\: M\xrightarrow{*} N \land M \xrightarrow{*}L \Rightarrow \exists P. \:\:M \xrightarrow{*} P \land L\xrightarrow{*} P$$
}

\cor{}{Se $=_\beta$ è la chiusura aimmetrica di $\xrightarrow{*}$ allora $M=_\beta N \Rightarrow \exists L. M\xrightarrow{*} L \land N \xrightarrow{*} L$}

\dfn{Booleani}{Si possono definire $\text{\underline{true}} \equiv \lambda x\:\:y.x$\footnote{Combinatore K} e $\text{\underline{false}} \equiv \lambda x\:\:y.y$\footnote{Combinatore O}

Partendo da ciò: $\text{\underline{if-then-else}} \equiv \lambda x\:\:y\:\:z.x\:\: y\:\: z$
}

\nt{Questa scrittura è basata sulla logica combinatorica, ma non è esattamente lo stesso nel $\lambda$-calcolo. Per essere precisi: tutti i modelli del $\lambda$-calcolo sono modelli della logica combinatorica, ma non il viceversa}

\ex{}{
\begin{itemize}
    \item if-then-else \underline{true} $M\:\:N \br \text{\underline{true}} M\:\:N \br M$;
    \item if-then-else \underline{false} $M\:\:N \br \text{\underline{false}} M\:\:N \br N$;
\end{itemize}
}

\section{Numerali di Church}

\dfn{Numerale di Church}{
$$\underline{n}\equiv \lambda x\:\:y.x(...(x\:\:y)...)$$

La $y$ si comporta come lo zero, mentre la $x$ come il successore
}

\ex{}{
$$\underline{0} \equiv \lambda x\:\:y.y$$

$$\underline{2} \equiv \lambda x\:\:y.x(x Y)$$

$$\underline{3} \equiv \lambda x\:\:y.x(x(x\:\:y))$$
}

\nt{In ogni numerale sono presenti $n \:\: x$ dove $n$ rappresenta il "numero" in decimale}

\dfn{Successore di un numerale}{
$$\text{\underline{succ }} n =_\beta n + 1 \equiv \lambda x\:\:y.x(x(...(x\:\:y)...))$$

$$\underline{n}\:\:x\:\:y =\beta x(...(x\:\:y)...)$$

Dunque \underline{succ }$\equiv \lambda z\:\:x\:\:y.x(z\:\:y\:\:x)$
}

\ex{}{
$$\underline{\text{succ 2}} = \lambda x\:\:y.x(\underline{2}\:\:x\:\:y)$$

$$= \lambda x\:\:y.x(x(x\:\:y)) \text{ , perchè } \underline{2} \:\:x\:\:y = x(x\:\:y)$$

$$= \underline{3} $$
}

\dfn{Somma}{
$$\underline{\text{add }} \:\underline{n}\:\:\underline{m} = \underline{n + m}$$

$$\underline{n + m} = \underline{\text{succ}}^n\:\:\underline{m}\equiv\underline{\text{succ}}(...(\underline{\text{succ}}\:\:\underline{m})...)$$

$$= \underline{n}\:\:\underline{\text{succ}}\:\:\underline{m}$$

Allora $\underline{\text{add }} \equiv \lambda x\:\:y.x\:\:\underline{\text{succ}}\:\:y$
}

\nt{Nello stesso modo si può definire \underline{mult} come iterazione di \underline{add}}

\dfn{Test per zero}{
$$\underline{\text{is-zero}}\:\:0 = \underline{\text{true}}$$

$$\underline{\text{is-zero}}\:\:n + 1 = \underline{\text{false}}$$

Allora \underline{is-zero} $\equiv \lambda n.n(\lambda z.\underline{\text{false}})\:\:\underline{\text{true}}$
}

\ex{}{
$$\underline{0}\:\:x\:\:y = y\:\:\:\:\:\:\:\:\:\:\:\:y\equiv \underline{\text{true}}$$

$$\underline{1}\:\:x\:\:y = x\:\: y\:\:\:\:\:\:\:\:\:\:\:\:x\equiv \lambda z.\underline{\text{false}}$$

$$\underline{2}\:\:x\:\:y = x(x\:\:y)\:\:\:\:\:\:\:\:\:\:\:\:x\equiv \lambda z.\underline{\text{false}}$$
}

\dfn{Ricorsione}{

$$\begin{cases}
    \text{fact}\:\:\underline{0} = \underline{1}\\
    \text{fact}\:\:\underline{n + 1} = \underline{\text{mult}} (n + 1) (\text{fact} \underline{n})
\end{cases}$$

Supponiamo di aver definito \underline{pred} tale che:

\begin{itemize}
    \item \underline{pred}\; \underline{0} = \underline{0};
    \item \underline{pred}\; \underline{n + 1} = \underline{n}.
\end{itemize}

$$\text{F}\:\:\underline{\text{fact}}\:\:\underline{n} = \text{if-then-else} \:\:(\underline{\text{is-zero}}\:\:\underline{n})\:\:\underline{1}\:\:(\underline{\text{mult}}\:\:\underline{n}\:\:(\underline{\text{fact}} \:\:(\underline{\text{pred}}\:\:\underline{n})))$$

$$\text{F}\equiv \lambda f\:\: x. \text{if-then-else}......(f\:\:(\underline{\text{pred}}\:\:n))....$$

Si suppone l'esistenza di una funzione \underline{fix} F = F (fix F)\footnote{Punto fisso}, allora:

$$\underline{\text{fact}}\equiv \underline{\text{fix F}} \text{ allora F \underline{fact} = \underline{fact}}$$

$$\text{\underline{fact}\:\underline{n} = F \underline{fact}}\:\:n = ....\underline{\text{fact}}\:\:(\underline{\text{pred}}\:\:\underline{n})$$

$$= ...... (\text{F} \underline{\text{ fact}})(\underline{\text{pred}}\:\:\underline{n})$$
}

\nt{Le funzioni ricorsive sono comunque calcolabili a patto che siano composte da funzioni calcolabili}

\thm{Teorema del punto fisso}{
$$\forall F\:\: \exists\:\: X . F\:\:X = X$$
}

\begin{myproof}
    Leggiamo l'equaziobe alla rovescia, quindi:
    $$X = F \:\: X$$
    Proviamo che $X = W\:\:W$, allora:
    $$W\:\:W = F\:(W\:\:W)$$
    Allora $W\equiv \lambda w.F\:(w\:\:w)$ risolve la seconda equazione e dunque, anche la prima.
\end{myproof}

\dfn{Operatore a punto fisso ($Y$)}{
$$\text{fix }\equiv \lambda f.(\lambda n.f\:(x\:\:x)) (\lambda x.f\: (x\:\:x)))$$
Allora fix $F = (\lambda n.F\:(x\:\:x))(\lambda x. F\:(x\:\:x)) = F\:((\lambda x.F\:(x\:\:x))(\lambda x.F\:(x\:\:x))) = F(\text{fix } F)$
}

\nt{Il $\lambda$-calcolo non tipato puro non è SN}

\thm{Teorema di Kleensn}{
Per ogni funzione calcolabile parziale esiste $F \:\in A$ tale che:
$$f\:(n_1,\:...,\:n_k) \simeq m \Leftrightarrow F\:\:n_1 .... n_k \br \underline{n}$$
Dove $f(n^\rightarrow)\simeq m$ significa che $f(n^\rightarrow)$ è definita uguale a $(n^\rightarrow = n_1,\:...,\:n_k)$
}









