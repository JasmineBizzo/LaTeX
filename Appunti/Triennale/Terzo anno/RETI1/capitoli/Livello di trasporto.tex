\chapter{Livello di trasporto}

\section{Principì del controllo della congestione}

\subsection{Scenario 1}

\subsection{Scenario 2}

\subsection{Scenario 3}

\subsection{Riflessioni e approcci}

\begin{itemize}
    \item [$\Rightarrow$] Il throughput non può mai superare la capacità aspettata;
    \item [$\Rightarrow$] Il ritardo aumenta quando ci si avvicina alla capacità;
    \item [$\Rightarrow$] Le perdite di pacchetti e le ritrasmissioni diminuiscono il throughput;
    \item [$\Rightarrow$] Gli eventuali duplicati diminuiscono ulteriolmente il throughput.
\end{itemize}

\dfn{Controllo End-End}{Non sono presenti feedback dal Network. Le congestioni possono essere inferite dalle perdite e dai ritardi}

\nt{Questo approccio è più semplice ma più rozzo e impreciso}

\dfn{Controllo assegnato al network}{I routers offrono un feedback diretto agli host che inviano e ricevono}

\nt{Il controllo End-End è il più usato}

\section{Il controllo della congestione del TCP}

\dfn{TCP AIMD (Additive Increasing Multiplicative Decreasing}{L'approccio consiste nell'aumentare la velocità trasmissione e rallentare nel caso di congestione. Si aumenta la velocità di 1 se arriva con successo un pacchetto, se si ha una perdita la velocità dimezza}

\nt{Si dimezza per favorire un'equa divisione delle risorse tra più utenti. Tuttavia la crescità additiva riduce il tempo di ripresa, rendendo impreticabile l'utilizzo di AIMD}

\dfn{TCP Reno}{Il TCP Reno dimezza la velocità ogni qual volta riceve 3 ACK duplicati}

\dfn{TCP Tahoe}{}

\section{TCP Cubic}

\dfn{TCP Cubic}{Il TCP Cubic effettua una funzione cubica, molto più veloce di una funzione lineare. Serve sapere la distanza temporale (parametro K) al momento in cui si andrà a toccare l'origine ($W_{max}$) e quindi a rallentare la trasmissione}