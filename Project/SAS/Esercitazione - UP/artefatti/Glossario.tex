
\chapter{\textcolor{dkgreen}{Glossario}}


\definecolor{mgray}{rgb}{0.9, 0.9, 0.9}
\begin{center}
    \begin{tabular}{ || >{\columncolor{mgray}}p{4.3cm} | p{7cm} | p{4.3cm}||}
    \hline\hline
        \rowcolor{mgray}
    \textbf{\textcolor{blue}{Termine}}& \textbf{\textcolor{blue}{Descrizione}} & \textbf{\textcolor{blue}{Sinonimi}}\\ \hline
    \rowcolor{mgray}
    \textbf{\textcolor{red}{Attori}} & &\\\hline

    Chef & Stabilisce il \textbf{menù} per gli eventi e ne supervisiona la preparazione.
    & \\\hline
    Cliente & Colui che commissiona l’organizzazione di un \textbf{evento}
    & \\\hline
    
    Cuoco & Prepara il \textbf{cibo}.
    & \\\hline

    Organizzatore & La persona che gestisce il \textbf{personale} e gli \textbf{eventi}.
    & \\\hline

    Personale & La parola personale è usata in modo ambiguo nel testo per intendere talvolta tutti i dipendenti (chef, cuochi, personale di servizio), talvolta soltanto i dipendenti soggetti a turni (cuochi e personale di servizio). In presenza di questo termine consultare il cliente per disambiguare. & \\\hline
    
    Personale di Servizio & Le persone (maître e camerieri) che si occupano del servizio durante l’evento stesso. & Staff di supporto \\\hline

    Utente & Generico utilizzatore dell’applicazione (ha necessariamente un ruolo fra \textbf{organizzatore}, \textbf{chef}, \textbf{cuoco}, \textbf{personale di servizio}). & \\\hline
    \rowcolor{mgray}
    \textcolor{red}{\textbf{Dominio dell'applicazione}} & & \\\hline

    \hline

    \end{tabular}
\end{center}