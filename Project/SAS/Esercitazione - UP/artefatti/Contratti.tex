\chapter{Contratti}

\paragraph{Pre-condizione generale:} L'attore è identificato con un'istanza \textit{ch} di Chef. 

\section{Passo 1}

\section*{1. generaFoglioRiepilogativo(\underline{evento}: Evento, \underline{servizio}: Servizio):}

\paragraph{Pre-condizioni:}

\begin{itemize}
  \item \underline{\textit{servizio}} \textbf{è previsto da} \underline{\textit{evento}};
  \item \underline{\textit{evento}} \textbf{è assegnato a} \underline{\textit{ch}};
  \item il Menù $m$ \textbf{è in uso per} \underline{\textit{servizio}}.
\end{itemize}

\paragraph{Post-condizioni:}

\begin{itemize}
  \item è stata creata un'istanza $f$ di FoglioRiepilogativo;
  \item è stata creata una lista $mc$ di MansioneDiCucina;
  \item $f$ \textbf{riferito a un} \underline{\textit{servizio}};
  \item \textlangle $ $ per ogni istanza di MansioneDiCucina $mc$ \textbf{presente in} $m$ \textrangle:
    \begin{itemize}
      \item è stata creata un'istanza di Compito $c$;
      \item $c.completato$ = sì;
      \item $c.fuoriMenu$ = no;
      \item $f$ \textbf{contiene} $c$;
      \item $c$ \textbf{consiste in} $mc$.
    \end{itemize}
  \item $ch$ \textbf{ha creato} $f$. 
\end{itemize}

\section*{1a.1. apriFoglio(\underline{evento}: Evento, \underline{servizio}: Servizio):}

\paragraph{Pre-condizioni:}

\begin{itemize}
  \item un FoglioRiepilogativo $f$ \textbf{è riferito a un} \textit{servizio};
  \item \textit{servizio} \textbf{è previsto da} \textit{evento}.
\end{itemize}

\paragraph{Post-condizioni:}

\begin{itemize}
  \item se $ch$ è assegnato a \textit{evento} e \textit{s.stato} == confermato.
\end{itemize}

\section{Passo 2}

\section*{2. aggiungiPrepRic(\underline{prepRic}: Mansione di cucina):}

\paragraph{Pre-condizioni:}

\begin{itemize}
  \item è in corso la modifica del FoglioRiepilogativo $f$.
\end{itemize}

\paragraph{Post-condizioni:}

\begin{itemize}
  \item è stata creata l'associazione \textbf{elencata} tra $f$ e \underline{\textit{prepRic}};
  \item è stata creata un'istanza di Compito $c$;
  \item $c.daPreparare$ = sì;
  \item $c.completato$ = no;
  \item $c.fuoriMenu$ = sì;
  \item $f$ \textbf{contiene} $c$;
  \item $c$ \textbf{consiste in} $mc$.
\end{itemize}

\section*{2a.1. rimuoviPrepRic(\underline{prepRic}: Mansione di cucina):}

\paragraph{Pre-condizioni:}

\begin{itemize}
  \item è in corso la modifica del FoglioRiepilogativo $f$;
  \item il Compito $c$ è \textbf{contenuto in} $f$ e \textbf{consiste in} \underline{\textit{prepRic}}.    
\end{itemize}

\paragraph{Post-condizioni:}

\begin{itemize}
  \item l'associazione \textbf{elencata} tra $f$ e \underline{\textit{prepRic}} è rimossa.
\end{itemize}

\section{Passo 3}

\section*{3. ordinaElenco(\underline{prepRic}: Mansione di cucina, \underline{posizione}: numerico):}

\paragraph{Pre-condizioni:}

\begin{itemize}
  \item è in corso la modifica del FoglioRiepilogativo $f$;
\end{itemize}

\paragraph{Post-condizioni:}

\begin{itemize}
  \item l'associazione \textbf{elencata} tra $f$ e \underline{\textit{prepRic}} è modificata in accordo alla nuova \underline{\textit{posizione}}.
\end{itemize}

\section{Passo 4}

\section*{4. consultaTabellone():}

\paragraph{Pre-condizioni:}

\begin{itemize}
  \item --------------
\end{itemize}

\paragraph{Post-condizioni:}

\begin{itemize}
  \item --------------
\end{itemize}

\nt{Trattandosi di un'interrogazione di sistema non ha pre-condizioni o post-condizioni perché può essere fatta in qualsiasi momento.}

\section{Passo 5}

\section*{5. assegnaCompito(\underline{compito}: Compito, \underline{turno}: TurnoDiCucina, \underline{cuoco?}: Cuoco):}

\paragraph{Pre-condizioni:}

\begin{itemize}
  \item è in corso la modifica del FoglioRiepilogativo $f$;
  \item $f$ \textbf{contiene} \underline{\textit{compito}};
  \item \underline{\textit{turno}} non è concluso.
\end{itemize}

\paragraph{Post-condizioni:}

\begin{itemize}
  \item \underline{\textit{compito}}.turno = \underline{\textit{turno}};
  \item $[$se è specificato \underline{\textit{cuoco}} e \underline{\textit{cuoco}}.disponibile == sì $]$ \underline{\textit{compito}}.cuoco = \underline{\textit{cuoco}}.
\end{itemize}

\section*{5a.1. modificaCompito(\underline{compito}: Compito, \underline{prepRic?}: Mansione di cucina, \underline{turno?}: TurnoDiCucina, \underline{cuoco?}: Cuoco):}

\paragraph{Pre-condizioni:}

\begin{itemize}
  \item è in corso la modifica del FoglioRiepilogativo $f$;
  \item $f$ \textbf{contiene} \underline{\textit{compito}};
  \item $[$se è specificato \underline{\textit{turno}}$]$ \underline{\textit{turno}} non è concluso.
\end{itemize}

\paragraph{Post-condizioni:}

\begin{itemize}
  \item $[$se è specificata \underline{\textit{prepRic}}$]$ \underline{\textit{compito}}.prepRic = \underline{\textit{prepRic}};
  \item $[$se è specificato \underline{\textit{turno}}$]$ \underline{\textit{compito}}.turno = \underline{\textit{turno}};
  \item $[$se è specificato \underline{\textit{cuoco}} e \underline{\textit{cuoco}}.disponibile == sì$]$ \underline{\textit{compito}}.cuoco = \underline{\textit{cuoco}};
\end{itemize}

\section*{5b.1. eliminaCompito(\underline{compito}: Compito)}

\paragraph{Pre-condizioni:}

\begin{itemize}
  \item è in corso la modifica del FoglioRiepilogativo $f$;
  \item $f$ \textbf{contiene} \underline{\textit{compito}}.
\end{itemize}

\paragraph{Post-condizioni:}

\begin{itemize}
  \item $[$se $turno$ \textbf{in cui è svolto} \underline{\textit{compito}} non è concluso$]$ 
\begin{itemize}
  \item l'associazione \textbf{contiene} tra $f$ e \underline{\textit{compito}} è eliminata;
  \item l'istanza \underline{\textit{compito}} è eliminata.
\end{itemize}
\end{itemize}

\section*{5c.1. specificaPronto(\underline{compito}: Compito)}

\paragraph{Pre-condizioni:}

\begin{itemize}
  \item è in corso la modifica del FoglioRiepilogativo $f$;
  \item $f$ \textbf{contiene} \underline{\textit{compito}}.
\end{itemize}

\paragraph{Post-condizioni:}

\begin{itemize}
  \item \underline{\textit{compito}}.completato = sì.
\end{itemize}

\section{Passo 6}

\section*{6. stima(\underline{compito}: Compito, \underline{tempoRichiesto?}: numerico, \underline{quantPorz?}: testo)}

\paragraph{Pre-condizioni:}

\begin{itemize}
  \item è in corso la modifica del FoglioRiepilogativo $f$;
\item \underline{\textit{compito}} \textbf{è svolto in} un TurnoDiCucina $t$ non concluso.
\end{itemize}

\paragraph{Post-condizioni:}

\begin{itemize}
  \item $[$se è specificato \underline{\textit{tempoRichiesto}}$]$ \underline{\textit{compito}}.tempoStimato = \underline{\textit{tempoRichiesto}};
  $[$se è specificato \underline{\textit{quantPorz}}$]$ \underline{\textit{quantità}}.tempoStimato = \underline{\textit{quantPorz}}.
\end{itemize}

