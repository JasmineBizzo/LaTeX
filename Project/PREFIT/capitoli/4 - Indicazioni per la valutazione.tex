\chapter{Indicazioni per la valutazione}

\section{Valutazione durante le lezioni/attività}

\section{Riguardo gli esercizi di programmazione...}

\subsection{Block Model}

\subsection{Misconceptions}

\section{Rubrica valutativà}

Questa piccola rubrica valutativa può essere utilizzata nell'ambito della fase di esercitazione in Haskell
come linea guida di valutazione.

\definecolor{mgray}{rgb}{0.9, 0.9, 0.9}

\begin{center}
    \begin{tabular}{ || >{\columncolor{mgray}}p{4.5cm} | >{\columncolor{RedPastel}}p{3.5cm} | >{\columncolor{BluePastel}}p{3.5cm} | >{\columncolor{GreenPastel}}p{3.5cm} ||}
    \hline\hline
        \rowcolor{lightgray}
    & \textbf{\textcolor{red}{Assente}} & \textbf{\textcolor{blue}{Parziale}} & \textbf{\textcolor{green}{Adeguata}}\\ \hline
        \textbf{Comprensione teorica} & Conoscenza della teoria gravemente insufficiente e/o assente.
        & Conoscenza della teoria sufficiente.
        & Conoscenza della teoria completa e/o approfondita.
        \\\hline

        \textbf{Sintassi di Haskell} & Mancanza di comprensione della sintassi di Haskell.
        & Comprensione della sintassi base di Haskell (variabili, guardie, etc.), difficoltà
        con concetti come "Inferenza di tipo" e/o "Pattern matching".
        & Ottima padronanza di Haskell e di GHCI.
        \\\hline

        \textbf{Esercizi} & Esercizi non svolti e/o svolti in maniera scorretta. 
        & Esercizi svolti in maniera parzialmente corretta, ma con qualche lacuna.
        & Esercizi svolti correttamente rispetto alle consegne.
        \\\hline

        \textbf{Padronanza della terminologia tecnica} & Terminologia assente e/o totalmente inadeguata. 
        & Terminologia sostanzialmente corretta, ma con alcune imprecisioni.
        & Terminologia pienamente corretta.
        \\\hline

        \textbf{Comprensione della ricorsione} & Assenza del concetto di ricorsione e/o idea completamente sbagliata della ricorsività.
        & Comprensione basilare della ricorsione.
        & Comprensione eccellente della ricorsione e delle sue implicazioni.
        \\\hline

    \hline
    \end{tabular}
\end{center}