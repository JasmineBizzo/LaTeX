\chapter{Guida per gli insegnanti}

\section{Consigli sull'utilizzo del materiale didattico}

\section{Snodi e indicatori per fasi}

\subsection{Attività 1}

\paragraph{\evidence{Fase 1:}}

\begin{itemize}
    \item Snodi:
    \begin{itemize}
        \item [$\Rightarrow$] Capire la struttura di una funzione ricorsiva;
        \item [$\Rightarrow$] Analizzare un programma ricorsivo;
        \item [$\Rightarrow$] Comprensione della sintassi di Haskell.
    \end{itemize}
    \item Indicatori:
    \begin{itemize}
        \item [$\Rightarrow$] Quali sono le caratteristiche di una funzione ricorsiva?
        \item [$\Rightarrow$] Identificate il passo base e il passo ricorsivo nel programma
                              "Fibonacci ricorsivo".
        \item [$\Rightarrow$] Elencare i principali tipi di Haskell.
    \end{itemize}
\end{itemize}

\subsection{Attività 2}

\paragraph{\evidence{Fase 1:}}

\begin{itemize}
    \item Snodi:
    \begin{itemize}
        \item [$\Rightarrow$]
    \end{itemize}
    \item Indicatori:
    \begin{itemize}
        \item [$\Rightarrow$]
    \end{itemize}
\end{itemize}

\subsection{Attività 3}

\paragraph{\evidence{Fase 1:}}

\begin{itemize}
    \item Snodi:
    \begin{itemize}
        \item [$\Rightarrow$]
    \end{itemize}
    \item Indicatori:
    \begin{itemize}
        \item [$\Rightarrow$]
    \end{itemize}
\end{itemize}