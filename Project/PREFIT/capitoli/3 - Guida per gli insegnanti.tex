\chapter{Guida per gli insegnanti}

\section{Snodi e indicatori per fasi}

\clm{}{}{Come specificato più volte nel documento gli "Indicatori" sono strutturati
in modo da essere posti direttamente, agli studenti e alle studentesse, sotto forma di quesiti.}

\subsection{Attività 1}

\paragraph{\evidence{Fase 1:}}

\begin{itemize}
    \item Snodi:
    \begin{itemize}
        \item [$\Rightarrow$] Capire la struttura di una funzione ricorsiva.
        \item [$\Rightarrow$] Analizzare un programma ricorsivo.
        \item [$\Rightarrow$] Comprensione della sintassi di Haskell.
    \end{itemize}
    \item Indicatori:
    \begin{itemize}
        \item [$\Rightarrow$] Quali sono le caratteristiche di una funzione ricorsiva?
        \item [$\Rightarrow$] Identificate il passo base e il passo ricorsivo nel programma
                              "Fibonacci ricorsivo".
        \item [$\Rightarrow$] Elencare i principali tipi di Haskell.
    \end{itemize}
\end{itemize}

\subsection{Attività 2}

\paragraph{\evidence{Fase 1:}}

\begin{itemize}
    \item Snodi:
    \begin{itemize}
        \item [$\Rightarrow$] Capire cosa simboleggia il risultato ottenuto eseguendo le istruzioni sulle matrioske.
        \item [$\Rightarrow$] Mettere in corrispondenza il concetto iterativo di "contatore" con il risultato
        ricorsivo ottenuto durante l'attività.
        \item [$\Rightarrow$] Capire l'importanza del passo base.
        \item [$\Rightarrow$] Comprensione dell'esistenza di soluzioni più "appropriate" per la risoluzione di problemi.
    \end{itemize}
    \item Indicatori:
    \begin{itemize}
        \item [$\Rightarrow$] Che risultato si è ottenuto? Cosa rappresenta? Soluzione: si è ottenuto il numero di matrioske "annidate" l'una dentro l'altra.
        \item [$\Rightarrow$] Era possibile raggiungere un simile risultato in un altro modo? Se sì, quale? Soluzione: si potevano aprire le matrioske e contarle una a una.
        \item [$\Rightarrow$] Che cosa succede se, quando rimonto una matrioska, non ne inserisco una (esclusa quella più interna) e riprovo a fare l'intera attività? Soluzione: il numero ottenuto questa volta sarà diminuito di "1".
        \item [$\Rightarrow$] Che cosa succede se, quando rimonto una matrioska, non inserisco quella più interna e riprovo a fare l'intera attività? Soluzione: in mancanza di un passo base il processo fallisce (dato che le matrioske sono finite),
         ma in alcune situazioni computazionali si potrebbe arrivare ad avere una ricorsione infinita. 
    \end{itemize}
\end{itemize}

\paragraph{\fancyglitter{Fase 2:}}

\begin{itemize}
    \item Snodi:
    \begin{itemize}
        \item [$\Rightarrow$] Simulare l'esecuzione di un programma ricorsivo e saperne predire l'output.
        \item [$\Rightarrow$] Comprendere l'esistenza di strutture dati che riferiscono sé stesse (Matrioska).
    \end{itemize}
    \item Indicatori:
    \begin{itemize}
        \item [$\Rightarrow$] Qual è l'output del programma se alla funzione \texttt{costruisciMatrioska} 
        passo 3 invece che 5? E se invece passo 7?
        \item [$\Rightarrow$] Cosa succede se passo 0 alla funzione \texttt{costruisciMatrioska}? Soluzione: l'output è "Nessuna Matrioska".
    \end{itemize}
\end{itemize}

\subsection{Attività 3}

\paragraph{\evidence{Fase 1:}}

\begin{itemize}
    \item Snodi:
    \begin{itemize}
        \item [$\Rightarrow$] Saper utilizzare forma infissa e forma prefissa di un'applicazione di funzione.
        \item [$\Rightarrow$] Saper scrivere una funzione con più argomenti.
        \item [$\Rightarrow$] Riconocere la maggiore leggibilità di una funzione che utilizza le guardie rispetto alla stessa funzione che utilizza un \texttt{IF}.
        \item [$\Rightarrow$] Conoscere le principali liste che si possono definire ([0..n], [0, 2..], [1, 3..], [a..b]).
        \item [$\Rightarrow$] Conoscere le principali funzioni che operano sulle liste.
        \item [$\Rightarrow$] Utilizzare correttamente la ricorsione per risolvere problemi.
    \end{itemize}
    \item Indicatori:
    \begin{itemize}
        \item [$\Rightarrow$] Quali sono i due modi di scrivere \texttt{div}? Soluzione: notazione infissa (con i backtick ``) e prefissa.
        \item [$\Rightarrow$] Come si scrive la "firma" di una funzione che prende due interi e restituisce una stringa? Soluzione: \texttt{Int -> Int -> String}.
        \item [$\Rightarrow$] Come si genera una lista che va da 4 a 9? Soluzione: [4..9].
    \end{itemize}
\end{itemize}


\paragraph{\fancyglitter{Fase 2:}}

\begin{itemize}
    \item Snodi:
    \begin{itemize}
        \item [$\Rightarrow$] Comprensione della struttura ricorsiva alla base degli alberi.
        \item [$\Rightarrow$] Osservare che alcune funzioni possono essere riutilizzate facilmente (tmax, tmin) in altri contesti e favoriscono una soluzione elegante e leggibile.
        \item [$\Rightarrow$] Saper riconoscere un algoritmo riguardante gli alberi in Haskell.
    \end{itemize}
    \item Indicatori:
    \begin{itemize}
        \item [$\Rightarrow$] Come viene effettuata una visita di un albero con la funzione \texttt{elements} e cosa restituisce?
        \item [$\Rightarrow$] Quale funzione serve per capire se un albero è di ricerca e su quali altre funzioni si basa?
        \item [$\Rightarrow$] Qual è il parametro della funzione \texttt{successore}?
    \end{itemize}
\end{itemize}
