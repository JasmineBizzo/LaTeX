\chapter{Introduzione}

\section{Livello di scuola, classe e indirizzo}

\qs{}{A chi è rivolta questa attività?}

\sol{A studenti del quarto/quinto anno di una scuola secondaria di secondo grado
di un indirizzo \textbf{scientifico}\footnote{Liceo Scientifico Op. 
Scienze Applicate}.}

\qs{}{Può essere adattata/rivolta a studenti di diverse età e indirizzi?}

\sol{Quest'attività può essere somministrata a studenti al quinto anno di superiori
senza alcuna modifica. Può, altresì, essere eseguita da studenti di età 
inferiore con alcune correzioni (che indicherò nel documento).}

\section{Motivazioni e finalità}

\qs{}{Perchè si è scelta proprio quest'attività?}

\sol{L'attività nasce dalla necessità di insegnare ai ragazzi
a ragionare in un modo diverso da quello a cui sono abituati.
Spesso gli studenti, quando si trovano davanti a un problema che
non sanno affrontare, tendono a procedere per tentativi ed errori,
ma ciò non è possibile con un linguaggio funzionale. Ciò porta gli
studenti a ingegnarsi per trovare una soluzione al problema,
che sia adatta ad essere espressa in un modo ricorsivo. La ricorsione
è un concetto che spesso viene trascurato nelle scuole superiori,
in quanto si tende a prediligere uno stile imperativo o a oggetti.
}

\section{Prerequisiti}

\begin{itemize}
    \item [$\Rightarrow$] Utilizzo di base del computer;
    \item [$\Rightarrow$] Concetto di algoritmo;
    \item [$\Rightarrow$] Concetto di variabile;
    \item [$\Rightarrow$] Logica booleana;
    \item [$\Rightarrow$] Propensione al ragionamento astratto.
\end{itemize}

\section{Contenuti}

I contenuti che presentano un asterisco blu (\textcolor{blue}{*}) sono
adatti anche a studenti di altri indirizzi o di età inferiore (quarta superiore). I contenuti che presentano un asterisco rosso (\textcolor{red}{*}) sono
opzionali per via della maggiore difficoltà, ma se gli alunni reagiscono bene al resto dell'attività
si potrebbero considerare come "bonus".

\begin{itemize}
    \item [$\Rightarrow$] \textcolor{blue}{*} Tipi primitivi di Haskell;
    \item [$\Rightarrow$] \textcolor{blue}{*} Funzioni;
    \item [$\Rightarrow$] \textcolor{blue}{*} Ricorsione;
    \item [$\Rightarrow$] Inferenza di tipo;
    \item [$\Rightarrow$] Funzioni a più argomenti;
    \item [$\Rightarrow$] Funzioni con guardie;
    \item [$\Rightarrow$] Liste;
    \item [$\Rightarrow$] Pattern matching;
    \item [$\Rightarrow$] \textcolor{red}{*} Funzioni anonime;
    \item [$\Rightarrow$] \textcolor{red}{*} Alberi.
\end{itemize}
\section{Traguardi e obiettivi}

\subsection{Obiettivi di apprendimento}

% fare tabella per organizzare in traguardi di competenze
% e obiettivi di conoscenze e abilità

\nt{I seguenti obiettivi di apprendimento sono stati scritti usando la tassonomia
di Bloom rivisitata e sono divisi in traguardi di competenze e obiettivi di conoscenze.
Il segno \{\textbf{+}\} indica un obiettivo procedurale, il segno \{\textbf{++}\}
indica un obiettivo concettuale e il segno \{\textbf{-}\} indica un obiettivo
metacognitivo.}

\begin{center}
    \begin{tabular}{ || p{8cm} | p{8cm} ||}
    \hline\hline

    \textbf{Traguardi di competenze} & \textbf{Obiettivi di conoscenze e abilità} \\ \hline
        
        \{+\} Implementare semplici algoritmi in Haskell. & \{+\} Saper utilizzare ghci. \\\hline
        \{++\} Capire l'utilizzo di un determinato costrutto. & \{++\} Conoscere la sintassi di Haskell. \\\hline
        \{-\} Costruire una funzione Haskell per risolvere un problema generico. & \{++\} Utilizzare, in modo opportuno, i costrutti. \\\hline
        \{++\} Analizzare un codice che utilizza la ricorsion. & \{++\} Valutare quando convenga utilizzare l'if e quando convenga usare le guardie.\\\hline
        \{+\} Saper scrivere un programma in modo che sia immediatamente
        comprensibile da altre persone.& \\\hline
    \hline
    \end{tabular}
\end{center}

\section{Materiali e strumenti necessari}

\begin{itemize}
    \item [$\Rightarrow$] Lavagna;
    \item [$\Rightarrow$] Gesso;
    \item [$\Rightarrow$] Computer con installato Haskell/ghci.
\end{itemize}

\section{Linguaggio}

\paragraph{Linguaggio scelto:} \evidence{Haskell}.

\qs{}{Perchè si è scelto questo linguaggio?}

\sol{Haskell è un linguaggio funzionale puro, il che lo rende ideale
per insegnare agli studenti un concetto importante come la ricorsione.
Inoltre Haskell presenta una sintassi molto semplice e pulita, che
non distrae gli studenti dal concetto che si sta insegnando: i codici risultano
molto più compatti rispetto ad altri linguaggi (C, Java, ecc...).
Infine, Haskell è un linguaggio unico nel suo genere, in quanto
"lazy" e "strongly typed", il che lo rende un linguaggio molto interessante
da studiare e da approfondire.}